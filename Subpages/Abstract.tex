% for Computer Society papers, we must declare the abstract and index terms
% PRIOR to the title within the \IEEEtitleabstractindextext IEEEtran
% command as these need to go into the title area created by \maketitle.
% As a general rule, do not put math, special symbols or citations
% in the abstract or keywords.
\IEEEtitleabstractindextext{%
	\begin{abstract}
	Der vorliegende wissenschaftliche Artikel befasst sich mit Microsofts Azure Plattform und dem darauf angebotenen Dienst zur Echtzeit-Eventverarbeitung - Azure Stream Analytics. Die Arbeit konzentriert sich dabei auf die Rolle des Service innerhalb der Plattform und welche Funktionen Azure Stream Analytics bietet. Für eine kritische Auseinandersetzung mit dieser Technologie wurden die offiziellen Angaben von Microsoft herangezogen, ausgewertet und mit der kosten- und lizenzfreien Alternative Apache Storm verglichen. Die Arbeit beinhaltet dabei eine Auswertung der Unterschiede dieser empirischen Untersuchung, welche unter anderem die Skalierbarkeit und Verfügbarkeit der Systeme betrachtet. Unter Berücksichtigung dieser Ergebnisse erfolgt eine Bewertung. Beide Plattformen haben die Vorteile einer PaaS-Lösung. Der Vergleich zeigt, dass beide dieselbe Kernarchitektur verwenden. Lediglich die Nutzergruppen sind hier zu unterscheiden: Stream Analytics zielt auf Administratoren ab, die unter Windows arbeiten und kaum Programmierkenntnisse besitzen wohingegen Apache Storm fundierte Programmierkenntnisse erfordert. Diese Arbeit ist für Leser interessant, welche sich oberflächlich mit dieser Technologie auseinandersetzen möchten. Es werden dabei keine tiefgreifenden Informationen geliefert.
	\end{abstract}
	
	% Note that keywords are not normally used for peerreview papers.
	\begin{IEEEkeywords}
		Microsoft, Azure, Stream Analytics, Real-Time, Event-Processing, Apache Storm, journal
\end{IEEEkeywords}}

