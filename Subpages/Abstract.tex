% for Computer Society papers, we must declare the abstract and index terms
% PRIOR to the title within the \IEEEtitleabstractindextext IEEEtran
% command as these need to go into the title area created by \maketitle.
% As a general rule, do not put math, special symbols or citations
% in the abstract or keywords.
\IEEEtitleabstractindextext{%
	\begin{abstract}
	Der vorliegende wissenschaftliche Artikel befasst sich mit Microsofts Cloud-Plattform Azure und dem darauf angebotenen Dienst zur Echtzeit-Eventverarbeitung - Azure Stream Analytics. Die Arbeit konzentriert sich auf die Rolle des Service innerhalb der Plattform und dessen verfügbaren Leistungsumfang. Für eine kritische Auseinandersetzung mit dieser Technologie werden die offiziellen Angaben von Microsoft herangezogen, ausgewertet und mit der kosten- und lizenzfreien Alternative Apache Storm verglichen. Die Arbeit beinhaltet dabei eine Auswertung der Unterschiede dieser Untersuchung, welche unter anderem die Skalierbarkeit und Verfügbarkeit der Systeme betrachtet. Unter Berücksichtigung dieser Ergebnisse erfolgt eine Bewertung. Beide Plattformen bieten die Vorteile einer Platform-as-a-Service-Lösung. Der Vergleich zeigt, dass beide Technologien dieselbe Kernarchitektur verwenden. Lediglich der Initialaufwand ist zu unterscheiden. Stream Analytics zielt auf Administratoren ab, die unter Windows arbeiten und kaum Programmierexpertise besitzen bzw. diese gar nicht benötigen, wohingegen Apache Storm fundierte Kenntnisse erfordert, um es aufzusetzen. Diese Arbeit zielt auf Leser ab, welche sich mit dem Azure-Dienst auseinandersetzen möchten. Es werden hier keine tiefgreifenden Informationen geliefert.
	\end{abstract}
	
	% Note that keywords are not normally used for peerreview papers.
	\begin{IEEEkeywords}
		Microsoft, Azure, Stream Analytics, Real-Time, Event-Processing, Apache Storm, journal
\end{IEEEkeywords}}

