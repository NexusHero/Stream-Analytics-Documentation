\section{Stream Analytics und Apache Storm im Vergleich}
In diesem Abschnitt werden die beiden Analyseplattformen verglichen. 


\subsection{Lizenzmodell}
Stream Analytics ist ein kommerzielles Produkt, welches nur innerhalb Microsoft Azure-Plattform genutzt werden kann. Hier kommt das Pay-As-You-go-Geschäftsmodell zum Einsatz, welches den Nutzern die Möglichkeit gibt, ausschließlich für den verursachten Datenverkehr zu bezahlen \cite{Pricing.2017}.  Desto höher das Datenvolumina ist, desto teurer wird die Nutzung. Damit können sowohl Kleinunternehmen als auch Großunternehmen dieses Produkt erwerben. Hingegen Apache Storm steht unter der Apache-Lizenz und ist somit kostenlos nutzbar \cite{lizenz.2004}. 

\subsection{Skalierbarkeit}
Da Azure sämtliche Hardware und Ressourcen bereitstellt, bestehen für die Nutzer keine speziellen Hardwareanforderungen. Welche Anforderung Stream Analytics an die genutzte Cloud-Infrastruktur hat, bleibt Microsoft vorbehalten und ist nicht bekannt \cite{samacha.2017}. Die Skalierung wird aus seitens Microsoft automatisch geregelt, sodass ein Event-Verkehr von 1 Gigabyte pro Sekunde verarbeitet werden kann \cite{samacha.19.12.2017b}. Apache Storm ist dagegen eine Standalone-Applikation und kann daher auf eigenen Systemen betrieben werden. Laut Apache kann mit 24 Gigabyte Arbeitsspeicher und 2-Intel-Prozessoren können 100 Million Byte-Messages verarbeitet werden \cite{apachescale.2017}. Auch hier sind keine konkreten Hardwarebedingungen gegeben, sondern sind abhängig von dem Anwendungsfall. Dies ist hängt von der Anzahl der Knoten ab \cite{samacha.2017}.


\subsection{Verfügbarkeit}
Beide Analyse-Plattformen versprechen hier eine Verfügbarkeit von 99.9 Prozent. Dazu kommt Wiederherstellungsmöglichkeiten, falls unvorhergesehen Fehler auftreten sollten. Apache Storm weist ebenfalls eine Verfügbarkeit von 99.9 Prozent auf, wobei der Benutzer für die Sicherheit der Daten verantwortlich ist. \cite{samacha.2017}. 

\subsection{Entwicklungsumgebung}
Bei Stream Analytics gibt es zwei Ansätze einen Stream Analytics-Job zu erstellen. Die erste Möglichkeit wird über das Azure-Portal geregelt. Hier kann der Benutzer über eine Benutzeroberfläche einen Job erstellen \cite{jeffstokes72.19.12.2017}. Die zweite Möglichkeit geht über eine Rest-Schnittstelle. Hier können durch Software, REST-Aufrufe Jobs erstellt, automatisiert und re-/konfiguriert werden \cite{samacha.19.12.2017}. 

\subsection{Abfragesprachen} \label{absprache}
Im Zusammenhang mit Stream Analytics stellt Microsoft eine SQL ähnliche Sprache zur Verfügung, mit der Abfragen vereinfacht werden soll. T-SQL definiert eine neue deklaravative Big-Data-Sprache aus dem Hause Microsoft, das ein Resultat aus der Kombination von ebenfalls Microsoft’s internentwickelte Sprache ''SCOPE'' und eine SQL ähnliche Sprache ist (Nicht ANSI SQL) \cite{sql.2016}. Hierdurch sind auch temporale Operatoren, welche Fensteraggregate und temporale Joins, unterstützten. Apache Storm unterstützt Abfragesprachen nicht standardmäßig. Hierfür muss der Anwender Programmcode entwickeln, das Programmierexpertise voraussetzt. Auch Fensterfunktionen werden standardmäßig nicht unterstützt und müssen ebenfalls implementiert werden \cite{samacha.2017}. Allerdings kann Apache Storm durch Einsatz von Fremdbibliotheken an Abfragesprachen, wie zum Beispiel ''Esper'' erweitert werden \cite{esper.2016}.

\subsection{Debugging-Unterstützung} \label{Debugging}
Zum Debuggen stehen Protokolle zur Verfügung, welches den Auftragsstatus und Auftragsvorgängen aufzeigen. Jedoch können Benutzer diese Debugg-Inhalte anpassen, sodass der Nutzer sein personalisiertes Logdatei erstellen kann. Auf Seitens wird das nicht automatisch protokolliert. Jedoch liegt das vollständig in der Hand des Entwicklers und das Logging frei gestalten \cite{apachedebugging.2106}.

\subsection{Datenformate}
Stream Analytics beschränkt sich auf die Datenformate Avro, JSON und CSV. Apache Storm dahingegen können beliebige Datenformate genutzt werden, welches eine dynamische Entwicklung und Anpassung möglich macht \cite{Klein.2017}. 

\subsection{Zielgruppe}
Mit Stream Analytics zielt Microsoft auf eine bestimmte Zielgruppe ab: Analysten. Nicht jeder verfügt über die nötigen Programmierkenntnisse um ein CEP-System aufzusetzen. Durch Stream Analytics wird das nicht mehr vorausgesetzt. Dieser Fakt geht aus den vorherigen Abschnitten \ref{Debugging} und \ref{absprache} heraus. Hingegen Apache Storm setzt auf eine andere Zielgruppe ab. Diese richtet sich auf Softwareentwickler mit fundierten Softwarekenntnissen \cite{Familiar.2017}. 
