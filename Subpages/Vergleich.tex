\section{Stream Analytics und Apache Storm im Vergleich}
Die beiden Technologien als Produkt lassen keinen direkten Vergleich zu. Azure Stream Analytics stellt bereits eine funktionsfähige Software dar, die über das Portal in Form eines Dienstes bezogen wird. Apache Storm hingegen muss vorab Aufgesetzt werden und beinhaltet keine Features wie eine SQL-ähnliche Abfragesprache und spezielle Funktionen für die Ereignisverarbeitung. In diesem Abschnitt werden jedoch einige Aspekte aufgeführt, die sich gegenüberstellen lassen. 

\subsection{Lizenzmodell und Kosten}
Azure Stream Analytics ist ein kommerzieller Dienst, welcher nur innerhalb des Azure-Portals genutzt werden kann. Hier kommt das Pay-As-You-go-Geschäftsmodell zum Einsatz, welches den Nutzern die Möglichkeit gibt, ausschließlich für den verursachten Datenverkehr zu bezahlen \cite{Pricing.2017}. Desto höher das Datenvolumina, desto teurer wird die Nutzung. Allerdings steht bei Azure nicht nur das Kontingent der verarbeiteten Daten im Vordergrund. Die bereits erwähnten Suits können auch über ein monatliches Abonnement erworben werden. Damit können sowohl Kleinunternehmen als auch Großunternehmen dieses Produkt nutzen. Microsoft hat zudem die Cortana Analytics Suite für Unternehmen jeder Größe entwickelt. Kleinere Unternehmen können davon profitieren, dass keine Investitionen oder teure Software-Lizenzen für den Start benötigt werden. Größere Unternehmen können von der Skalierbarkeit der Azure Cloud-Plattform profitieren \cite{Azure.2017}. Apache Storm hingegen steht unter der Apache-Lizenz und ist somit kostenlos nutzbar \cite{lizenz.2004}.

\subsection{Skalierbarkeit und Hardwareanforderung}
Da Azure sämtliche Hardwareressourcen bereitstellt, bestehen für die Nutzer keine speziellen Hardwareanforderungen. Welche Anforderung Stream Analytics an die genutzte Cloud-Infrastruktur zugrunde legt, bleibt Microsoft vorbehalten und ist nicht bekannt \cite{samacha.2017}. Die Skalierung wird seitens Microsoft automatisch geregelt, sodass ein Event-Verkehr von einem Gigabyte pro Sekunde verarbeitet werden kann \cite{samacha.19.12.2017b}. Apache Storm dagegen ist eine Standalone-Applikation und kann daher auf eigenen Systemen betrieben werden. Laut Apache kann mit 24 Gigabyte Arbeitsspeicher und 2-Intel-Prozessoren ein Kontingent von 100 Million Byte-Messages verarbeitet werden \cite{apachescale.2017}. Auch hier sind keine konkreten Hardwarebedingungen gegeben. Diese sind vom Anwendungsfall und der Anzahl Knoten abhänig \cite{samacha.2017}.

\subsection{Verfügbarkeit}
Beide Analyse-Plattformen versprechen hier eine Verfügbarkeit von 99.9 Prozent. Für eine Hochverfügbarkeit bietet Azure eine Wiederherstellungsmöglichkeit. Bei Apache Storm  trägt der Benutzer die Verantwortung der Sicherheit seiner Daten \cite{samacha.2017} \cite{apachescale.2017}. 

\subsection{Entwicklungsumgebung}
Stream Analytics unterstützt zwei Ansätze, einen Auftrag zu erstellen: Azure-Portal und REST. Im ersten Fall kann der Benutzer über eine entsprechende Benutzeroberfläche einen Job erstellen \cite{jeffstokes72.19.12.2017}. Die zweite Möglichkeit geht über eine REST-Schnittstelle, welche über Quellcode angesprochen wird. Somit ist es möglich via eigener Implementierung Jobs zu erstellen, diese zu automatisieren und zu re-/konfigurieren \cite{samacha.19.12.2017}. Apache Storm erfordert ein eigenverantwortliches Aufsetzen des gesamten Systems \cite{apache.2017}. 

\subsection{Abfragesprachen} \label{absprache}
Wie bereits erwähnt stellt Microsoft eine SQL-ähnliche Sprache zur Verfügung, mit welcher Abfragen erfolgen. In Kombination mit Scope definiert ASQL eine neue deklaravative Big-Data-Sprache \cite{sql.2016}. Somit werden temporale Operatoren unterstützt. In Apache Storm sind Abfragesprachen nicht standardmäßig enthalten. Hierfür muss der Anwender Programmcode schreiben, was Programmierkenntnisse voraussetzt. Auch Fensterfunktionen werden nicht unterstützt und erfordern eine eigene Implementierung \cite{samacha.2017}. Allerdings kann Apache Storm durch Einsatz von Fremdbibliotheken an Abfragesprachen, wie zum Beispiel ''Esper'' erweitert werden \cite{esper.2016}.

\subsection{Debugging-Unterstützung} \label{Debugging}
Zum Debuggen steht ein Protokoll zur Verfügung, welches den Auftragsstatus und die Auftragsvorgänge aufzeigt. Jedoch können Benutzer die Debugg-Inhalte nicht anpassen. Der Anwender kann keine personalisierte Logdatei erstellen. Auf Seitens Apache Storm gibt es kein automatisches Logging. Dies liegt vollständig in der Hand des Entwicklers. Er kann das Logging dafür frei gestalten \cite{apachedebugging.2106}.

\subsection{Datenformate}
Azure Stream Analytics beschränkt sich auf die Datenformate Avro, JSON und CSV. Apache Storm hingegen kann beliebige Datenformate nutzen. Das ermöglicht eine dynamischere Entwicklung \cite{Klein.2017}. 

