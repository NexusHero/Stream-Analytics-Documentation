\section{Vergleich}

In diesem Abschnitt werden die beiden Analyseplattformen verglichen. 


\subsection{Lizenzmodell}
Stream Analytics ist ein kostenpflichtiges Produkt, welches ausschließlich auf der Microsoft Azure-Plattform genutzt werden kann und nirgendwo anders. Hier kommt das Pay-As-You-go-Geschäftsmodell zum Einsatz, welches den Nutzern die Möglichkeit gibt, ausschließlich für den verursachten Eventverkehr zu bezahlen \cite{Pricing.2017}. Je mehr Events registriert werden, desto teurer wird die Nutzung. Hingegen Apache Storm steht unter der Apache-Lizenz und ist somit kostenlos \cite{lizenz.2004}. 

\subsection{Skalierbarkeit}
Da Stream Analytics innerhalb der Cloud-Plattform Azure genutzt werden kann, bestehen für die Nutzer keine speziellen Hardwareanforderungen. Welche Anforderung Stream Analytics an die genutzte Cloud-Infrastruktur hat, bleibt Microsoft vorbehalten und ist nicht bekannt \cite{samacha.2017}. Die Skalierung wird aus seitens Microsoft automatisch geregelt, sodass ein Event-Verkehr von 1 Gigabyte pro Sekunde verarbeitet werden kann \cite{samacha.19.12.2017b} Apache Storm ist dagegen eine Standalone-Applikation und kann daher auf eigenen Systemen betrieben werden. Laut Apache kann mit 24 Gigabyte Arbeitsspeicher und 2-Intel-Prozessoren können 100 Million Byte-Messages verarbeitet werden \cite{apachescale.2017}. Auch hier sind keine konkreten Hardwarebedingungen gegeben, sondern sind abhängig von dem Anwendungsfall.

\subsection{Abfragesprachen}
Im Zusammenhang mit Stream Analytics stellt Microsoft eine SQL ähnliche Sprache, mit der Abfragen vereinfacht werden soll. T-SQL definiert eine neue deklaravative Big-Data-Sprache aus dem Hause Microsoft, das ein Resultat aus der Kombination von ebenfalls Microsoft’s internentwickelte Sprache ''SCOPE'' und eine SQL ähnliche Sprache ist (Nicht ANSI SQL) \cite{sql.2016}. Hierdurch sind auch temporale Operatoren, welche Fensteraggregate und temporale Joins, unterstützten  Abfragesprachen sind standardmäßig nicht unterstützt. Hierfür muss der Anwender Programmcode entwickeln, die wiederum Programmierexpertise voraussetzt. Auch Fensterfunktionen werden standardmäßig nicht unterstützt und müssen ebenfalls implementiert werden. Dies trifft auch auf  \cite{samacha.2017}. Allerdings kann Apache Storm durch Einsatz von Fremdbibliotheken an Abfragesprachen, wie zum Beispiel ''Esper'' erweitert werden \cite{esper.2016}.


\subsection{Debugging-Unterstützung} 
Zum Debuggen stehen Protokolle zur Verfügung, welches den Auftragsstatus und Auftragsvorgängen aufzeigen. Jedoch können Benutzer diese Debugg-Inhalte anpassen, sodass der Nutzer sein personalisiertes Logdatei erstellen kann. Auf Seitens wird das nicht automatisch protokolliert. Jedoch liegt das vollständig in der Hand des Entwicklers und das Logging frei gestalten \cite{apachedebugging.2106}.


\subsection{Datenformate}
Stream Analytics beschränkt sich auf die Datenformate Avro, JSON und CSV. Apache Storm dahingegen können beliebige Datenformate genutzt werden, welches eine dynamische Entwicklung und Anpassung möglich macht \cite{Klein.2017}. 

\subsection{Zielgruppe}

