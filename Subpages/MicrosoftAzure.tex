\section{Microsoft Azure}
 Azure ist Microsofts Hybrid-Cloud-Plattform, die seit 2010 verfügbar ist. Dies bedeutet, dass Daten sowie Apps lokal und in der Cloud gespeichert und verknüpft werden können. Azure enthält eine stetig wachsende Anzahl an Tools, Diensten, Anwendungen und SQL- sowie NSQL-Datenbanken, die dem Nutzer über das Portal angeboten werden. Zudem bieten sie umfassende Dienste für künstliche Intelligenz, mit der die Entwicklung umfangreicher, intelligenter Anwendungen ermöglicht wird.  Nutzer können über Azure auf Hochleistungsgrafikprozessoren zugreifen, wodurch ihnen die Entwicklung im Bereich autonomes Fahren, Spracherkennung sowie Daten- und Videoanalyse erlaubt wird. Neben umfangreicher Prozessorleistung stellt Microsoft auch Speicherplatz zur Verfügung. Somit erlaubt Azure das sichere Speichern eigener Daten in der Cloud \cite{Sicherheit.2017b}. Des Weiteren möchte Microsoft damit die IT in Unternehmen flexibler gestalten, da sich mit Cloud-Computing Infrastrukturen in Form von virtuellen Servern oder Load Balancern digital zur Verfügung stellen lassen. Stichwörter hier sind Infrastructure-, Platform- und Application-as-a-Service. Auch Lösungen unter Einbeziehung von beliebige Entwicklungstools oder -sprachen werden dem Anwender ermöglicht \cite{Klein.2017}. \\ \\
Grundlage der Azure-Plattform ist ein wachsendes Netzwerk aus verteilten Rechenzentren. Somit kann eine enorm hohe Verfügbarkeit gewährleistet werden. Die Sicherung der eigenen Anwendungen und Daten wird durch diese Hochverfügbarkeit und eine Notfallwiederherstellung gewährleistet. Durch kontinuierliche Investitionen in die neueste Infrastrukturtechnologie, bleibt Microsofts Plattform zuverlässig, kosteneffizient und umweltverträglich \cite{Klein.2017}.\\ \\
Derzeit liegt Microsoft hinter dem Marktführer Amazon, welches in \ref{fig:statistik} illustriert wird. Auf der offiziellen Website von Azure wird die Frage gestellt ‚Azure oder AWS?‘. Microsoft gibt darauf eine prägnante Antwort: ‚Azure ist die richtige Wahl‘ \cite{Microsoft.2017}.
\begin{figure}[ht!]
	\centering
	\includegraphics[width=1.0\linewidth]{images/statistik}
	\caption{Marktanteil nach Umsatz \cite{Statista.2017b}} %Generelle
	\label{fig:statistik}
\end{figure}
\\ \\ Azure und AWS sind beides solide Plattformen. Beide haben ihre Vor- und Nachteile. Welche sich tatsächlich besser eignet, hängt von der Anforderung ab \cite{PeterTsai.2016}. Was Azure allerdings besonders zugute kommt, ist die ergonomische Benutzeroberfläche zum Aufsetzen komplexer Systeme. Dies ist durch das Verknüpfen verschiedener Dienste möglich. Die Schritt-für-Schritt-Anleitungen, die dem Anwender zur Verfügung gestellt werden, zeigen, dass die Einrichtung entsprechender Dienste, Server und Cloud-Instanzen mit wenig Aufwand durchzuführen ist. Somit macht es den Übergang zur Cloud-basierten Infrastruktur für viele Unternehmen unbeschwerter. Vor allem das nahtlose Zusammenarbeiten der angebotenen Dienste bzw. Azure-Instanzen erlaubt es, eine robuste Infrastruktur aufzubauen. Als Exempel kann hier der IoT Hub und die sich daran anschließende Ereignisverarbeitung Stream Analytics dargelegt werden \cite{PeterTsai.2016}.\\ \\
Eine Azure-Instanz besteht dabei aus einem virtuellen Server. Dieser wird über Hyper-V (Microsofts Hypervisor zur Visualisierung ihrer Server \cite{searchdatacenter.2017}) betrieben. Kosten fallen nach Verwendung eigener Cloud-Instanzen an. Bei Stream Analytics steht das Kontingent der verarbeiteten Daten im Vordergrund \cite{PeterTsai.2016}.  
