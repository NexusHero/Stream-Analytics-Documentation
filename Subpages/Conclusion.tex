\section{Conclusion}
Für Unternehmen sehr attraktiv, da die angebotene Produktvielfalt ein sehr breites Spektrum abdeckt. Es werden keine tiefgreifenden Kenntnisse im Programmieren benötigt werden, also keine Entwickler. Zudem fällt das Aufsetzen von bestimmten Umgebungen vollständig weg. Azure kann einfach verwendet werden und die Dienste lassen sich einfach miteinander verknüpft. Der Support ist erfahrungsgemäß auch sehr gut, da jedem Nutzer ein entsprechender Ansprechpartner zugewiesen wird, an den man sich wenden kann.\\ \\Allerdings sind die Kosten für die angebotenen Dienste recht hoch. Es wird zwar angepriesen, dass alles sehr kosteneffizient gelöst sei und diese sind, wenn man den Umfang der Leistungen betrachtet, sicherlich fair, doch für kleine Unternehmen sind diese doch recht hoch. Es wird einem hier auch nicht ermöglicht mal eben einen Testaccount aufzusetzen und alles zu probieren. Das anfängliche Guthaben von 170 Euro, welches 30 Tage lang für beliebige Azure-Produkte genutzt werden kann ist direkt aufgebraucht, sodass man keinen Eindruck vom System erhält. Es gibt die Möglichkeit Informationen seiner Kreditkarte für die Identitätsüberprüfung zu hinterlegen. Eine Alternative gibt es leider nicht. Damit hätte man die Möglichkeit über den IoT Hub theoretisch bis zu 8000 Nachrichten pro Tag zu generieren. Stream Analytics zählt allerdings nicht zu den kostenlosen Diensten, wodurch die Kombination nicht über einen aussagekräftigen Zeitraum getestet werden kann.\\ \\ Obwohl Stream Analytics bereits einige Zeit auf dem Markt ist, gibt es auch keinerlei wissenschaftliche Artikel, welche sich kritisch mit der Thematik auseinander setzen. Sämtliche Literatur beruht auf Aussagen von Microsoft Mitarbeitern.\\ \\Während beispielsweise das Oracle Event Processing auf Frameworks wie Spring und OSGi setzt, weiß man bei Azure nicht, was unter der Haube steckt. (Allerdings werden bei solchen Produkten Programmierkenntnisse vorausgesetzt, was bei Azure nicht der Fall ist und es somit einen bedeutenden Vorteil bietet.) Auch Oracle bietet mit ihrem Produkt eine Alternative. Einmal richtig aufgesetzt, erlaubt es auch Anwendern ohne Programmierkenntnisse eine entsprechende Analyse der Daten.\\ \\Es gibt bisher keinerlei Arbeiten darüber, die das System kritisch betrachten und bewerten. Es sind lediglich von Microsoft zur Verfügung gestellte Beispiele und Dokumentationen zu finden. Aufgrund der stark begrenzten Literatur bzw. der limitierten Quellen, ist davon auszugehen, dass die Technologie noch nicht besonders oft genutzt wird. Es gibt sie seit 2014, doch andere Technologien sind offensichtlich beliebter.



% Can use something like this to put references on a page
% by themselves when using endfloat and the captionsoff option.
\ifCLASSOPTIONcaptionsoff
  \newpage
\fi