\section{Fazit}
Storm genießt einen großen Einfluss in der Big-Data-Community und ist äußerst skalierbar, fehlertolerant und kann in bestehende Technologien integriert werden. Große Namen wie Twitter greifen darauf zurück, um ihre Daten in Echtzeit zu analysieren. Allerdings ist die Einrichtung von Apache Storm keinesfalls trivial. Dies ist der große Vorteil von Azure Stream Analytics, welches im Grunde keine Einrichtung und/ oder Wartung erfordert, da es von Microsoft als Dienst (Software-as-a-Service) angeboten wird. Nicht jeder Anwender verfügt über fundamentale Programmierkenntnisse, um ein CEP-System aufzusetzen. Daher ist Stream Analytics ein profitables Produkt für Unternehmen, die einen umfangreichen initialen Aufwand nicht stemmen wollen oder können. Dies mag unter anderem an mangelndem Expertenwissen liegen. Das Einrichten eines Jobs beschränkt sich hier auf wenige Klicks im Azure-Portal. In ihm werden sämtliche Ein- und Ausgaben konfiguriert sowie entsprechende Abfragen ausgeführt. Auch Azure Stream Analytics ist sehr skalierbar. Ein einzelner Job kann bereits Millionen Ereignisse pro Sekunde verarbeiten. Da es mit anderen Diensten auf der Azure-Plattform angeboten wird, erlaubt es außerdem ein nahtloses Kombinieren verschiedener Azure-Services. Dazu zählen die in diesem Artikel aufgelisteten Produkte (Azure Storage, Azure SQL Database und Microsoft Power BI). Durch die von SQL abgeleitete Abfragesprache können Nutzer bereits eine bekannte Sprache verwenden, um die Daten zu analysieren oder persistent zu speichern usw. Es zielt zunehmend auf Analysten ab, da keine Programmierexpertise vorausgesetzt wird. Zudem fällt das Aufsetzen von bestimmten Umgebungen vollständig weg. Jedem Nutzer wird dabei ein Ansprechpartner delegiert.\\ Obwohl Stream Analytics bereits einige Zeit auf dem Markt ist, gibt es keinerlei wissenschaftliche Artikel, welche sich kritisch mit der Thematik auseinandersetzen. Es sind lediglich von Microsoft zur Verfügung gestellte Literatur, Beispiele und Dokumentationen zu finden. Aufgrund der stark begrenzten Quellen, war es leider nicht möglich tiefgreifende Einblicke zu erhalten. Künftig wird die Entwicklung dieses Dienstes weiterhin verfolgt. Der bisherige Mangel an wissenschaftlichen Artikeln könnte sich noch ändern, da schließlich immer mehr Geräte Teil einer IoT-Lösung werden und somit immer mehr Daten das Internet förmlich überfluten und analysiert werden müssen. Interessant bleibt auch die Entwicklung in Verbindung mit künstlicher Intelligenz bzw. maschinellem Lernen. Dies bietet nämlich eine mächtige Erweiterung zum Erkennen von Anomalien etc.
 

%

% Can use something like this to put references on a page
% by themselves when using endfloat and the captionsoff option.
\ifCLASSOPTIONcaptionsoff
  \newpage
\fi