\section{Diskussion}
Storm genießt einen großen Einfluss in der Big-Data-Community und ist äußerst skalierbar, fehlertolerant und kann in bestehende Technologien integriert werden. Große Namen wie Twitter greifen darauf zurück, um ihre Datenmengen in Echtzeit zu verarbeiten. Allerdings ist die Einrichtung von Apache Storm keinesfalls trivial. Anfängliche Investitionen in Hardware und womöglich Software können anfallen. Dies ist der große Vorteil von Azure Stream Analytics, welches im Grunde keine Einrichtung und/ oder Wartung erfordert, da es von Microsoft als Dienst (Software-as-a-Service) angeboten wird. Nicht jeder Anwender verfügt über fundamentale Programmierkenntnisse. Zumal Storm an sich keine CEP Engine bietet. Hier muss ein weiteres Produkt wie Esper mit eingebunden werden, um auch komplexe Ereignisse über Regelwerke und Interpreter zu erkennen und zu verarbeiten. Dies ist bei Azure Stream Analytics nicht der Fall ist. Die Engine steckt bereits im Dienst. Daher ist Azure Stream Analytics ein profitables Produkt für Unternehmen, die einen umfangreichen initialen Aufwand nicht stemmen wollen oder können. Dies mag unter anderem an mangelndem Expertenwissen liegen. Das Einrichten eines Azure Stream Analytics-Jobs beschränkt sich hier auf wenige Klicks im Azure-Portal. In ihm werden sämtliche Ein- und Ausgaben konfiguriert sowie entsprechende Abfragen ausgeführt. Auf die Skalierbarkeit wird bei Azures Diensten viel Wert gelegt. Ein einzelner Job kann bereits Millionen Ereignisse pro Sekunde verarbeiten. Da es mit anderen Diensten auf der Azure-Plattform angeboten wird, erlaubt es außerdem ein nahtloses Kombinieren verschiedener Azure-Services. Dazu zählen die in diesem Artikel aufgelisteten Produkte (Azure Storage, Azure SQL Database und Microsoft Power BI). Wie aus der Arbeit hervorging, werden auch einige, häufig vorkommende Szenarien als vorkonfigurierte Lösung angeboten um die anfänglichen Hürden noch weiter zu reduzieren.\\ \\
Durch die von SQL abgeleitete Abfragesprache können Nutzer bereits eine bekannte Sprache verwenden, um die Daten zu analysieren oder persistent zu speichern usw. Es zielt zunehmend auf Analysten ab, da keine Programmierexpertise vorausgesetzt wird. Zudem fällt das Aufsetzen von bestimmten Umgebungen vollständig weg. Jedem Nutzer wird dabei ein Ansprechpartner delegiert.\\ \\
Während bei Storm die Verarbeitungslogik mit Spouts und Bolts umgesetzt wird, weiß man bei Azure Stream Analytics nicht direkt, wie die Daten intern verarbeitet werden. Einiges deutet darauf hin, dass auch Azure Stream Analytics ähnlich vorgeht. Logisch betrachtet bleibt auch nur ein solcher Ansatz. Entsprechende Agents werden hier die Ausführungseinheiten darstellen. Für die Echtzeitverarbeitung der Daten wird auch bei Microsofts Lösung auf die Lambda-Architektur gebaut. Die Untersuchungen ergaben, dass die Kappa-Architektur trotz potentiell schnellerer Verarbeitung keine Alternative darstellt. Für einige Szenarien ist es erforderlich, Daten über einen gewissen Zeitraum zwischenzuspeichern. Genaue Einblicke gewährt Microsoft hier nicht. Der Anwender muss allerdings auch nicht wissen, was unter der Haube steckt. Kein Software-Know-How ist erforderlich. Es ist möglich eine entsprechende Lösung zu konfigurieren, ohne zu wissen, wie sie intern funktioniert. Bei Storm hingegen braucht man umfangreiche Einblicke in die Technologie, da ansonsten das Implementieren einer Lösung unmöglich wäre.

\section{Ausblick}
Obwohl Azure Stream Analytics bereits einige Zeit auf dem Markt ist, gibt es keinerlei wissenschaftliche Artikel, welche sich kritisch mit der Thematik auseinandersetzen. Es sind lediglich von Microsoft zur Verfügung gestellte Literatur, Beispiele und Dokumentationen zu finden. Aufgrund der stark begrenzten Quellen, war es leider nicht möglich tiefgreifende Einblicke zu erhalten. Die offiziellen Angaben der Anbieter klingen sehr vielversprechend, doch eine Überprüfung dieser Werte steht noch aus. Auch eine Frage bleibt offen. Wie bereits erläutert, unterstützt Azure Stream Analytics drei Fensterfunktionen. Den Autoren stellt sich die Frage, warum drei unterschiedliche Zeitfenster angeboten werden, jedoch keine Längenfenster. Grundsätzlich reicht auch eine Zeitfensterfunktion, welche nur über den zeitlichen Versatz unterschiedlich konfiguriert werden könnte. \\ \\
Trotz dem nicht offengelegten Hintergrund bzgl. der präzisen Funktionsweise von Azure Stream Analytics sind die Autoren der Meinung, dass es eine gewinnbringende Plattform darstellt. Künftig wird die Entwicklung dieses Dienstes verfolgt. Der bisherige Mangel an wissenschaftlichen Artikeln könnte sich noch ändern, da schließlich immer mehr Geräte Teil einer IoT-Lösung werden und somit immer mehr Daten das Internet förmlich überfluten und analysiert werden müssen. Interessant bleibt auch die Entwicklung im Machine Learning-Bereich. Hier ergeben sich mächtige Erweiterungen zum Erkennen von Anomalien.\\



%http://www.soutier.de/blog/2014/02/23/lambda-architektur/
%http://www.soutier.de/blog/2017/01/29/kappa-architektur-und-no-etl/

% Can use something like this to put references on a page
% by themselves when using endfloat and the captionsoff option.
\ifCLASSOPTIONcaptionsoff
  \newpage
\fi