\IEEEraisesectionheading{\section{Einführung}\label{sec:introduction}}

\begin{quote} \textit{\glqq In the Internet of Things […] the real value will remain in the services behind the scenes processing terabytes of data to help people live better lives. \grqq~}\cite{Floarea.2014}\\ \end{quote} 

\IEEEPARstart{D}{ie} zunehmende Anzahl von allgegenwärtigen und leistungsfähigen, intelligenten Geräten wie beispielsweise Sensoren, Mobiltelefone und andere Geräte fördern laufend die Entwicklung von Anwendungen zur Datenstromanalyse, um Ereignisse zu überwachen. Die massiven Datenströme, die durch diese Anwendungen erzeugt werden, haben das Internet der Dinge (loT) zu einer wichtigen Quelle für Big Data gemacht [6]. Bereits heute sind über 20 Milliarden Geräte mit dem Internet verbunden und laut Prognosen sollen diese Zahlen bis in wenigen Jahren drastisch anwachsen \cite{Statista.2017}. Das damit Verbundene Wachstum der zu verarbeitenden (nicht-endlichen \cite{Mock.2005}) Datenströmen stellt eine zunehmende Herausforderung dar, nicht zuletzt, weil diese Daten in Echtzeit verarbeitet werden sollen und sie sich ständig in Bewegung befinden [6]. Ereignisgesteuerte Systeme, die einen Strom von Sensordaten auswerten, um relevante Informationen zu extrahieren, brauchen eine automatische Verarbeitung dieser Daten. Genau hier setzt Real-Time Stream Analytics an. Microsofts Cloud-baiserte Lösung leitet aus einem nicht endenden Strom aus Sensordaten Muster und komplexe Ereignisse ab, wodurch es einem System ermöglicht wird, auf Situationen entsprechende Maßnahmen anzuwenden und so geeignet darauf zu reagieren (Abfangen von kritische Situationen in der Produktion \cite{rcrwireless.2016}). Nicht nur Microsoft bietet im Bereich Complex Event Processing eine Lösung. Kosten- und lizenzfreie Produkte wie Apache Storm bieten hier eine Alternative. Laut Microsoft ist ihre Azure Plattform und die darauf angebotenen Dienste eine unglaublich gute Sache. In dieser Arbeit werden dabei diese beiden Technologien miteinander verglichen. Vorab wird kurz erläutert, wobei es sich um Complex Event Processing handelt. Dies ist für das weitere Verständnis der Arbeit wichtig. Anschließend erfolgt eine Beschreibung von Microsofts Cloud-Plattform. Auf diese baut das Kapitel auf, in welchem schließlich Azures Stream Analytics genauer beschrieben wird und welche Features die Technologie beinhaltet. Daraufhin erfolgt der Vergleich von Azures Stream Analytics und Apache Storm sowie die abschließende Bewertung. 