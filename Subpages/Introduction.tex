\IEEEraisesectionheading{\section{Einführung}\label{sec:introduction}}

\begin{quote} \textit{\glqq In the Internet of Things […] the real value will remain in the services behind the scenes processing terabytes of data to help people live better lives. \grqq~}\cite{Floarea.2014}\\ \end{quote} 

\IEEEPARstart{D}{er} Hype um den IoT-Bereich geht weiter und wächst kontinuierlich \cite{peter.2015}. Bereits 2017 sind weltweit 23.35 Milliarden Geräte internetfähig und diese Zahl soll - laut Prognosen - bis 2025 auf 74.44 Milliarden ansteigen \cite{Statista.2017}. Das bedeutet, dass die Menge zu verarbeitende Daten ebenfalls ansteigen, sodass eine Datenstromverarbeitung immer wichtiger wird. Datenströme stellen eine nicht-endliche Folge von Datensätzen dar \cite{Mock.2005}. Damit wachsen die Anforderungen im Kontext Skalierbarkeit für ein Complex Event Processing (CEP), die Verarbeitung möglichst in Echtzeit geschieht, sodass beispielsweise kritische Situationen in der Produktion abgefangen werden können \cite{rcrwireless.2016}. Hierfür existieren  bereits einige Datenstromanalyse-Software, welche mit Echtzeitverarbeitung werben: Apache Storm (Opensource-Projekt) \cite{apache.2017} und Microsoft's Stream Analytics(kommerzielles Produkt) \cite{Microsoft.2017}. Laut Microsoft ist ihre Azure Plattform und die darauf angebotenen Dienste eine unglaublich gute Sache. Mit diesem wissenschaftlichen Artikel wird Microsofts Dienst zur Echtzeit-Eventverarbeitung kritisch betrachtet. Er setzt sich mit der Frage auseinander, ab wann es sinnvoll ist auf Microsofts Stream Analytics zurückzugreifen und welche Unterschiede sich hier zur Konkurrenz herausarbeiten lassen. Dabei wird die Leistungsfähigkeit beider Technologien betrachtet …
