\IEEEraisesectionheading{\section{Einführung}\label{sec:introduction}}

\begin{quote} \textit{\glqq In the Internet of Things […] the real value will remain in the services behind the scenes processing terabytes of data to help people live better lives. \grqq~}\cite{Floarea.2014}\\ \end{quote} 

\IEEEPARstart{W}{ie} aus dem Zitat von Dr. Floarea Serban hervorgeht, handelt es sich beim Internet der Dinge vor allem um die Dienste zur Verarbeitung der massiven Datenströme. Die zunehmende Anzahl von leistungsfähigen und intelligenten Geräten wie beispielsweise Sensoren, Smartphones etc., fördern laufend die Entwicklung von Anwendungen zur Datenstromanalyse, um Ereignisse in Echtzeit zu überwachen. Bereits heute sind über 20 Milliarden Geräte mit dem Internet verbunden und laut Prognosen sollen diese Zahlen bis in wenigen Jahren drastisch anwachsen \cite{Statista.2017}. Das damit verbundene Wachstum der zu verarbeitenden (nicht-endlichen) Datenströme, stellt eine Herausforderung dar. Nicht zuletzt, weil diese Daten in Echtzeit verarbeitet werden sollen und sie sich ständig in Bewegung befinden \cite{Prosise.}.\\ \\  
Um relevante Informationen aus einem Strom von Sensordaten auszuwerten, brauchen ereignisgesteuerte Systeme eine Automatisierung dieser Datenverarbeitung. Genau hier setzt Real-Time Stream Analytics an. Microsofts Cloud-basierte Lösung leitet aus einem unendlichen Datenfluss Muster und komplexe Ereignisse ab, wodurch es einem solchen System ermöglicht wird, auf gewisse Gegebenheiten entsprechende Maßnahmen anzuwenden. So ist beispielsweise das Abfangen kritischer Situationen in der Produktion möglich und es können potenzielle Ausfälle von Maschinen rechtzeitig erkannt und vermieden werden \cite{rcrwireless.2016}. Jedoch offeriert nicht nur Microsoft eine Lösung im Bereich Complex Event Processing. Opensource-Produkte wie Apache Storm bieten hier eine Alternative. In dieser Arbeit werden diese beiden Technologien miteinander verglichen. Vorab wird kurz erläutert, wobei es sich um Complex Event Processing handelt. Dies ist für das weitere Verständnis der Arbeit wichtig. Anschließend erfolgt eine Beschreibung von Microsofts Cloud-Plattform. Auf diese baut das Kapitel auf, in welchem der Azure-Dienst Stream Analytics genauer beschrieben und auf dessen Features näher eingegangen wird. Daraufhin erfolgt der Vergleich beiden Echtzeitverarbeitungssystemen sowie eine abschließende Bewertung. Diese Arbeit zielt auf Leser ab, welche sich mit dem Azure-Dienst auseinandersetzen möchten. Es werden hier keine tiefgreifenden Informationen geliefert.\\ \\
Einen Überblick zum Thema liefert das Buch IoT Solutions in Microsoft's Azure IoT Suite \cite{Klein.2017}. Allerdings zeigt sich, dass es im Kontext von Azure Stream Analytics keine wissenschaftlichen Arbeiten von dritten gibt, die den Funktionsumfang dieser Technologie bezeugen und kritisch beanstanden. Weiterhin gilt es zu betonen, dass es sich bei Azure Stream Analytics um ein kommerzielles Produkt handelt, welches keine direkte Analyse über den Quellcode zulässt. Diese Arbeit stützt sich auf offizielle Literatur und Angaben von Microsoft \cite{Familiar.2017} \cite{Klein.2017}. Folglich können keine exemplarisch wichtigen Arbeiten vorgestellt werden, die mit dem Thema des Artikels in Beziehung stehen.