\IEEEraisesectionheading{\section{Einführung}\label{sec:introduction}}

\begin{quote} \textit{\glqq In the Internet of Things […] the real value will remain in the services behind the scenes processing terabytes of data to help people live better lives. \grqq~}\cite{Floarea.2014}\\ \end{quote} 

\IEEEPARstart{D}{ie} zunehmende Anzahl von leistungsfähigen und intelligenten Geräten wie beispielsweise Sensoren, Smartphones etc., fördern laufend die Entwicklung von Anwendungen zur Datenstromanalyse, um Ereignisse in Echtzeit zu überwachen. Die massiven Datenströme, die durch diese Gerätschaften generiert werden, haben das Internet der Dinge (IoT) zu einer wichtigen Quelle für Big Data gemacht \cite{Prosise.}. Bereits heute sind über 20 Milliarden Geräte mit dem Internet verbunden und laut Prognosen sollen diese Zahlen bis in wenigen Jahren drastisch anwachsen \cite{Statista.2017}. Das damit verbundene Wachstum der zu verarbeitenden (nicht-endlichen) Datenströme, stellt eine zunehmende Herausforderung dar. Nicht zuletzt, weil diese Daten in Echtzeit verarbeitet werden sollen und sie sich ständig in Bewegung befinden \cite{Prosise.}.\\ \\ Ereignisgesteuerte Systeme, die einen Strom von Sensordaten auswerten, um relevante Informationen zu extrahieren, brauchen eine automatische Verarbeitung dieser Daten. Genau hier setzt Real-Time Stream Analytics an. Microsofts Cloud-basierte Lösung leitet aus einem unendlichen Datenfluss Muster und komplexe Ereignisse ab, wodurch es einem solchen System ermöglicht wird, auf Situationen entsprechende Maßnahmen anzuwenden und so geeignet darauf zu reagieren. So ist beispielsweise das Abfangen kritischer Situationen in der Produktion möglich und es können potenzielle Ausfälle von Maschinen rechtzeitig erkannt und vermieden werden \cite{rcrwireless.2016}. Jedoch offeriert nicht nur Microsof eine Lösung im Bereich Complex Event Processing. Opensource-Produkte wie Apache Storm bieten hier eine Alternative. Laut Microsoft ist Azure und die darauf angebotenen Dienste eine unglaublich gute Sache. In dieser Arbeit werden diese beiden Technologien miteinander verglichen. Vorab wird kurz erläutert, wobei es sich um Complex Event Processing handelt. Dies ist für das weitere Verständnis der Arbeit wichtig. Anschließend erfolgt eine Beschreibung von Microsofts Cloud-Plattform. Auf diese baut das Kapitel auf, in welchem der Azure-Dienst Stream Analytics genauer beschrieben und auf dessen Features näher eingegangen wird. Daraufhin erfolgt der Vergleich beiden Echtzeitverarbeitungssystemen sowie eine abschließende Bewertung. 