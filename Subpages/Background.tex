\section{Stream Analytics und Complex Event Processing}
Stream Analytics beschreibt grundlegend das Analysieren und Verarbeiten von Datenströmen, allerdings zur Laufzeit und nahezu in Echtzeit. Das bedeutet, dass die Zeit zwischen der Datenaufnahme und dem Ergebnis möglichst gering ist (typischerweise in Sekunden gemessen). Grundsätzlich geht es bei der Verarbeitung um das Filtern, Aggregieren und Suchen von Mustern. Die Resultate lassen sich anschließend an entsprechende Anwendungen weiterleiten. Stream Analytics gehört somit zum Themenbereich Complex Event Processing \cite{GesellschaftfurInformatik.2009}.\\ \\ 
Complex Event Processing (CEP) ist ein Sammelbegriff. Dieser umfasst Methoden, Techniken und verschiedene Werkzeuge, mit denen voneinander unabhängige Ereignisse stetig und zeitnah verarbeitet werden. Um auf Ereignisse zu reagieren, werden aus dem eingehenden Datenstrom benötigte Informationen extrahiert. Typische Reaktionen wären das Senden von Benachrichtigungen, einfache Aktionen oder Interaktionen mit Geschäftsprozessen \cite{GesellschaftfurInformatik.2009}. \\ \\
Datenströme bestehen häufig zu einem Großteil aus irrelevanten Daten („noise“). Nur ein Bruchteil bilden dabei Nutzdaten („signal“). Ein Ereignisverarbeitungssystem filtert diese über geeignete Mechanismen heraus. Das ermöglicht eine Konzentration der relevanten Anteile, während beispielsweise Messfehler direkt aussortiert werden. Nachfolgend sind die Nutzdaten mit anderen Datenströmen in Bezug zu setzen. Durch diese Korrelation der einzelnen Sensordaten ergeben sich Ereignisse. Zusätzlich können die Daten mit Informationen (Meta-Daten) aus unterschiedlichen Datenbanken angereichert werden. Dies kann nützlich sein, um noch aussagekräftigere Ergebnisse zu erzielen. Dabei ist vor allem die zeitliche Komponente ein wichtiger Faktor. Für verschiedene Situationen ist es notwendig, dass Ereignisse in einem gewissen Zeitraum bzw. einer bestimmten Reihenfolge auftreten. Da Ereignisse, die in einem gewissen zeitlichen Abschnitt eingepflegt werden, von großer Bedeutung sein können, gibt es die Funktion der Zeitfenster. Abfragen auf einen unendlichen Ereignisstrom sind nur in bestimmten Ausschnitten denkbar \cite{GesellschaftfurInformatik.2009}. Nach dieser Vorverarbeitung des Ereignisstroms, lassen sich die Daten gezielt nach definierten Mustern absuchen. Werden entsprechende Muster gefunden, kann das System darauf reagieren \cite{Bruening.2016}.

% Can use something like this to put references on a page
% by themselves when using endfloat and the captionsoff option.
\ifCLASSOPTIONcaptionsoff
  \newpage
\fi